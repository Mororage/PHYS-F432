\RequirePackage{fix-cm}
\usepackage[utf8]{inputenc}
\usepackage[T2A,T1]{fontenc}
\usepackage[french]{babel}
\usepackage{csquotes}
\usepackage[top=2cm, bottom=2cm, left=1.5cm, right=1.5cm]{geometry}

% Mathématiques et symboles
\usepackage{amsmath, amsfonts, amssymb, amsthm, mathtools, bm}
%\usepackage{mdframed}% Permet d'encadrer les équations
\usepackage{braket}
\usepackage{cancel} %Barres diagonales
\usepackage{esvect}
\usepackage{centernot} % Pour barrer des symboles
%\usepackage{halloweenmath}
\usepackage{tensor}

% Graphiques et couleurs
\usepackage{graphicx}
\usepackage{float}
\usepackage[xcdraw, table]{xcolor}

% Bibliographie
\usepackage[backend=biber, style=numeric]{biblatex}
\addbibresource{refermance.bib} % Import de la bibliographie

% Autres fonctionnalités utiles
\usepackage{delarray}
\usepackage{parskip}
\usepackage{enumitem}
\setlist[itemize]{leftmargin=*, label=\textbullet}

\usepackage{hyperref}
\usepackage{fullpage}
\usepackage{caption}
\usepackage{subcaption}
\usepackage{lipsum}
\usepackage{listings}
%\usepackage{wasysym} % Astronomical symbols

\usepackage{musicography} % Nice breakers

\usepackage{lmodern} % To switch to Latin Modern
\rmfamily % To load Latin Modern Roman and enable the following NFSS declarations.
% Declare that Latin Modern Roman (lmr) should take
% its bold (b) and bold extended (bx) weight, and small capital (sc) shape, 
% from the corresponding Computer Modern Roman (cmr) font, for the T1 font encoding.
\DeclareFontShape{T1}{lmr}{b}{sc}{<->ssub*cmr/bx/sc}{}
\DeclareFontShape{T1}{lmr}{bx}{sc}{<->ssub*cmr/bx/sc}{}

% Encadrements
\usepackage[framemethod=tikz]{mdframed} % Pour lignes verticales sur les théorèmes

% En-têtes et pieds de page
\usepackage{fancyhdr}
\usepackage{lastpage}
\pagestyle{fancy}

% Réinitialisation des en-têtes et pieds de page
\fancyhead{}
\fancyfoot{}

% Configuration des en-têtes et pieds de page

\fancyhead[L]{Théorie de la Gravitation}
\fancyhead[R]{\nouppercase\leftmark}
\fancyfoot[C]{\footnotesize Page \thepage\ sur \pageref*{LastPage}}


% Correction de la hauteur des en-têtes pour éviter les erreurs
\setlength{\headheight}{14.5pt} % Hauteur de l'en-tête
\setlength{\topmargin}{-1.5cm} % Diminue la marge supérieure pour remonter l'en-tête
\setlength{\headsep}{1cm} % Espacement vertical entre l'en-tête et le texte

\setlength{\footskip}{1.5cm} % Distance entre le texte principal et le début du pied de page
\addtolength{\textheight}{+1cm} % Réduit la hauteur de la zone de texte pour libérer de l'espace en bas

\usepackage{color}
\definecolor{myblue}{rgb}{.8, .8, 1}
\usepackage{empheq}

\newlength\mytemplen
\newsavebox\mytempbox

\makeatletter
\newcommand\mybluebox{%
    \@ifnextchar[%]
       {\@mybluebox}%
       {\@mybluebox[0pt]}}

\def\@mybluebox[#1]{%
    \@ifnextchar[%]
       {\@@mybluebox[#1]}%
       {\@@mybluebox[#1][0pt]}}

\def\@@mybluebox[#1][#2]#3{
    \sbox\mytempbox{#3}%
    \mytemplen\ht\mytempbox
    \advance\mytemplen #1\relax
    \ht\mytempbox\mytemplen
    \mytemplen\dp\mytempbox
    \advance\mytemplen #2\relax
    \dp\mytempbox\mytemplen
    \colorbox{myblue}{\hspace{1em}\usebox{\mytempbox}\hspace{1em}}}