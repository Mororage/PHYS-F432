% Legacy
\renewcommand{\thesection}{\arabic{section}}

% Propositions de macros:
% 1. Raccourcis lettres (p.ex. R)
% 2. Raccourcis symboles de math (p.ex. () )
% 3. Autres


% 1. Raccourcis lettres
\newcommand{\td}{\mathrm{d}} % le petit d de dérivé totale
\newcommand{\pd}{\partial} % d rond de dérivé partielle
\newcommand{\R}{\mathbb{R}} % Le joli R réels
\newcommand{\N}{\mathbb{N}} % Le joli N naturels
\newcommand{\Z}{\mathbb{Z}} % Le joli Z entiers
\newcommand{\Q}{\mathbb{Q}} % Le joli Q rationnels
\renewcommand{\C}{\mathbb{C}} % Le joli courps C complexe
\newcommand{\K}{\mathbb{K}} % Le joli K (corps arbitraire)
\renewcommand{\H}{\mathrm{H}} % Un H droit pour l'Hamiltonien
\newcommand{\Hd}{\mathcal{H}} % Un H caligraphique (Hilbert)
\renewcommand{\L}{\mathrm{L}} % Un L droit (Lagrangien)
\newcommand{\Ld}{\mathcal{L}} % Densité lagrangienne
\renewcommand{\S}{\mathrm{S}} % Action
\newcommand{\F}{\mathcal{F}} % F caligraphique (usages variés)
\newcommand{\M}{\mathcal{M}}
\newcommand{\E}{\mathcal{E}} % E caligraphique (énergies, ...)
\renewcommand{\U}{\mathcal{U}} % U caligraphique (énergies, ...)
\newcommand{\cRM}[1]{\MakeUppercase{\romannumeral #1}}	% Chiffre romain en majuscule
\DeclareMathOperator{\sgn}{sgn}
%\newcommand{\Zhe}{\mbox{\usefont{T2A}{\rmdefault}{m}{n}\CYRZH}}
\newcommand{\Zhe}{\mathfrak{X}} %Champ de vecteurs
\newcommand{\RNum}[1]{\uppercase\expandafter{\romannumeral #1\relax}}

% Decorum break : 
\newcommand{\cutebreak}{\begin{center} \textcolor{gray}{\musSegno \quad \musSegno\quad \musSegno\quad \musSegno\quad \musSegno }\end{center}}

% 2. Raccourcis symboles

\newcommand{\desp}{\mathrm{d}^3x} % Element de vomlume infinit. d'espace
\newcommand{\dest}{\mathrm{d}^4x} % Volume infinit. d'espace-temps
\newcommand{\dtem}{\mathrm{dt}} % Volume infinit. de temps
\newcommand{\deq}{\equiv} % Défini égal notation
\newcommand{\vect}[1]{\bm{\mathrm{#1}}} % Notation vectorielle (gras+flèche)
\newcommand{\tindices}[2]{\indices{#1}{\!#2}}


\newcommand{\ltc}{\left\{} % parenthèses adaptées.
\newcommand{\rtc}{\right\}}
\newcommand{\ltb}{\left\[} % parenthèses adaptées.
\newcommand{\rtb}{\right\]}
\newcommand{\lt}{\left(} % parenthèses adaptées.
\newcommand{\rt}{\right)}
\newcommand{\atP}[1]{\left. #1 \right|_P}
\newcommand{\dt}{\frac{d}{dt}} % Dérivé totale
\newcommand{\vide}{\emptyset} % Le VIDE
\newcommand{\nc}{\framebox[1cm][c]{$\boldsymbol{\La}$}\quad} % Preuve: sens inverse
\renewcommand{\sc}{\framebox[1cm][c]{$\boldsymbol{\Ra}$}\quad} % Preuve: sens direct
\newcommand{\ser}[1]{\sum_{#1 = 0}^{\infty}} % Somme infinie sur N
\newcommand{\seri}[1]{\sum_{#1 = -\infty}^{\infty}} % Somme infinie sur Z
\newcommand{\st}{\boldsymbol{(\star)}} % Une étoile (pourquoi pas)
\newcommand{\serr}[1]{\sum_{#1 = 1}^{\infty}}
\newcommand{\purple}[1]{\textcolor{purple}{#1}}
\newcommand{\blue}[1]{\textcolor{blue}{#1}}
\newcommand{\gray}[1]{\textcolor{darkgray}{#1}}

% Autres
% Opérateurs
\DeclareMathOperator{\ISO}{ISO}
\DeclareMathOperator{\IO}{IO}


% Environnements théorème
\newmdenv[
  linecolor=black,
  linewidth=2pt,
  topline=false,
  bottomline=false,
  rightline=false,
  leftline=true,
  skipabove=5ptpt,
  skipbelow=0pt,
]{theoremframe}

\theoremstyle{plain}
\newtheorem{theorem}{Théorème}[chapter]
\newtheorem{lemme}[theorem]{Lemme}
\newtheorem{corol}[theorem]{Corollaire}
\newtheorem{prop}[theorem]{Proposition}
\newtheorem{law}[theorem]{Loi}
\newtheorem{post}[theorem]{Postulat}
\newtheorem{propri}[theorem]{Propriété}
\newtheorem{exmp}[theorem]{Exemple}
\newtheorem{obs}[theorem]{Observation}

\theoremstyle{definition}
\newtheorem{defi}[theorem]{Définition}
\newtheorem{rmk}[theorem]{Remarque}
\newtheorem{notat}[theorem]{Notation}
\newtheorem{rap}[theorem]{Rappel}
\newtheorem{expe}[theorem]{Expérience}
\newtheorem{exerc}[theorem]{Exercice}
\newtheorem{quest}[theorem]{Question}
