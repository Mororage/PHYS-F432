\section*{Préface}
\subsubsection{Préface de Moritz}
Le syllabus ci-présent a principalement été retranscrit des notes de cours manuscrites de Stéphane Detournay, principalement par Moritz Schnor et Victoria Spruyt. Lorsque nous avons commencé ce projet, le but était principalement d'avoir des notes dactylographiées. J'ai toutefois tenté de parfois aller légèrement au-délà du cours pour inviter le·a lecteur·ice à approfondir la matière ; ces sections seront clairement indiquées par un astérix ( * ). J'espère également que ce document reste vivant, et se complètera au fil des années. Je metterai en place un système de feedback pour que les étudiant·es puissent apporter des corrections ou suggestions. 



Le syllabus ci-présent est principalement une retranscription des notes de cours manuscrites de Stéphane Detournay. Le but est d'avoir une référence dactylographiée claire, suivant la matière du cours tout en permettant d'aller un peu plus loin que ce qui est vu explicitement en cours. Ainsi, j'ai rajouté quelques sections pour complémenter la matière vue, ou pour ajouter des précisions. Ces parties rajoutées seront la plupart du temps clairement indiquées, afin de garder la possibilité d'étudier strictement la matière du cours. 
\cutebreak
\subsubsection{Préface de Stéphane Detournay*}
Le but de ce cours est de fournir une introduction à la théorie de la \emph{Relativité Générale d'Einstein} (1915), qui représente notre description la plus moderne de l'\emph{interaction gravitationnelle}. C'est l'interaction avec laquelle nous sommes sans doute le plus familiers, mais qui demeure la moins bien comprise au niveau fondamental.\\

La relativité générale est l'une des théories les plus élégantes jamais élaborées, et permet d'aborder des sujets aussi fascinants que la physique des trous noirs ou la théorie du \emph{Big Bang}. La relativité générale n'est pas forcément un sujet compliqué, contrairement à la légende : elle est sans dout conceptuellement plus simple que l'autre pilier de la physique contemporaine, la théorie quantique des champs. Néanmoins, elle nécessitera de se familiariser avec certains outils mathématiques peut-être nouveaux (dont notamment la \emph{géométrie différentielle}) ainsi qu'avec des concepts parfois contre-intuitifs. Nous pouvons alors apprécier à la fois l'élégance de la théorie, mais aussi sa capacité à faire des prédictions expérimentales, qui sont jusqu'à ce jour vérfifiées de manière spectaculaire.\\

La relativité générale est également devenue incontournable ou preque en physique théorique moderne et constitue un domaine extrêmement actif de recherche, avec des ramifications et implications dans de nombreuses branches comme l'astrophysique, la cosmologie, la résolutions d'EDP\footnote{Équations aux Dérivées Partielles.}, théorèmes de singularités ou encore des comportements chaotiques. Plus récemment, des connexions plus inattendues sont apparues avec la physique des particules, de la matière condensée, la dynamique des fluides, les superconducteurs ou le problème de \emph{gauge-gravity correspondence}. De manière plus terre-à-terre, la relativité générale joue un rôle crucial dans certaines technologies familières comme le GPS. \\

L'idée fondamentale de la relativité générale est très simple : contrairement aux autres interactions de la nature (comme l'interaction électromagnétique par exemple) qui sont représentées par des champs définies sur l'espace-temps, la gravitation se manifeste par une courbure de l'espace-temps lui-même : la gravité est \emph{inhérente} à l'espace-temps. Le contenu en matière (et énergie-impulsion) dicte à l'espace-temps comment il doit se courber. En retour, des particules-test se déplacent, en présence de gravitation, librement (en ce, en absence de force externe) mais dans un espace-temps courbe. 
\begin{center}
    \textit{Matter tells space-time how to bend, space-time tells matter how to move.}
\end{center}
\begin{flushright}
    - John Archibald Wheeler
\end{flushright}




Une grande partie du cours sera consacré à développer les équations de champs d'Einstein: 
\begin{align}
    \label{eq:Einstein introduction}
    \boxed{G\indices{_\mu_\nu} = \kappa \, T\indices{_\mu _\nu}}
\end{align}
où 
\begin{itemize}
    \item $G\indices{_\mu _\nu}$ est le \emph{tenseur d'Einstein}, qui encode la géométrie de l'espace-temps,
    \item  $T\indices{_\mu _\nu}$ est le \emph{tenseur d'énergie-impulsion}, représentant le contenu en énergie (ou matière),
    \item $\kappa$ est la \emph{constante gravitationnelle de couplage d'Einstein}.
\end{itemize}
\cutebreak